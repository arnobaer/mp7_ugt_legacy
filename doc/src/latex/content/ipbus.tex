%------------------------------------------------------------------------------
%
% Repository path   : $HeadURL: https://hbergauer@forge.hephy.oeaw.ac.at/scm/svn/project-cmstrigger/GlobalTriggerUpgrade/doc/latex/gt-mp7-firmware-specification/content/ipbus.tex $
% Last committed    : $Revision: 4338 $
% Last changed by   : $Author: hbergauer $
% Last changed date : $Date: 2016-06-17 08:45:22 +0200 (Fri, 17 Jun 2016) $
% Description       : IPBus Structure: Fabric/Slave
% ------------------------------------------------------------------------------
\section {IPBus}\label{sec:ipbus}
\textbf{\textit{Under construction!!!}}

% For the Gigabit Ethernet links, MAC cores are instantiated. Depending on the compilation settings,
% one or two Gigabit Ethernet Links are instantiated (for bench-top and crate operation respectively).
% In the case of bench-top operation, the MAC core is configured as 1000Base-T in order to
% communicate with the external PHY. In the case of crate operation, both MAC cores are configured
% as 1000Base-X for interfacing with the Gigabit Ethernet Switch carried on th
% e crate’s MCH.
% For every MAC core, an IPbus endpoint is also instantiated. The IPBus system allows the control of
% hardware via a ‘virtual bus’, using a standard IP-over-gigabit-Ethernet network connection. The
% IPBus specifies a simple transaction protocol between the hardware and a software controller, which
% assumes an A32/D32 connection to slave devices connected to the hardware endpoint. The current
% IPbus firmware implementation is using a UDP/IP protocol and a simple synchronous SoC bus.
% 
% This protocol is based upon the Wish bone SoC protocol, and is compatible with Wishbone
% cores. However, there are two important differences:
% \begin{itemize}
%  \item The master is not required to explicitly deassert strobe between cycles. However, it is guaranteed to deassert strobe or begin the new cycle 
%  on the clock cycle following ack. 
% \item Slaves are not al lowed to tie ack high, and must deassert ack on the same clock cycle that strobe is
% deasserted. However, it is allowed to tie ack to strobe, if a zero-wait-state response is always possible.
% \end{itemize}
% 
% \subsection{Implementation}\label{sec:ad-impl}
% under construction ...
% 
% \subsection{Interface}\label{sec:ad-interface}
% 
% %\lstinputlisting[language=VHDL,caption=AD interface specification]{interfaces/ad.vhd} 
% 
% \subsection{How to use registers} \label{subsec:UseRegisters}
% \subsection{How to add Slaves/Fabric} \label{subsec:addRegisters}
% %The following document should be integrated in this chapter:
% %1) IPbus Network Architecture for μTCA Hardware:https://svnweb.cern.ch/trac/cactus/browser/trunk/doc/uHAL_etwork_addressing.docx
% %2) The IPbus Protocol: https://svnweb.cern.ch/trac/cactus/browser/trunk/doc/ipbus_protocol_v2_0.docx
